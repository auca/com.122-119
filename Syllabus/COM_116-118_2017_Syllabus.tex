\documentclass[12pt,a4paper,oneside]{article}

\usepackage[margin=3cm]{geometry}

\usepackage{hyperref}
\hypersetup{
    pdftitle={COM 116, 117, 118 Structural Programming, Programming I, II},%
    pdfauthor={Toksaitov Dmitrii Alexandrovich},%
    pdfsubject={Syllabus},%
    pdfkeywords={COM;}{116;}{117;}{118;}{syllabus;}{programming;}{I;}{II;}{Java},%
    colorlinks,%
    linkcolor=black,%
    citecolor=black,%
    filecolor=black,%
    urlcolor=black
}

\newcommand{\R}[1]{\uppercase\expandafter{\romannumeral #1\relax}}

\begin{document}

    \title{COM 116, 117, 118, Structural Programming, Programming \R{1}, \R{2}}
    \author{
        American University of Central Asia\\
        Department of Software Engineering
    }
    \date{}
    \maketitle

    \section{Course Information}

        \begin{description}
            \item[Course ID]\hfill\\
                COM 116, 2967\\
                COM 117, 2968\\
                COM 118, 4322
            \item[Course Repository]\hfill\\
                \url{https://github.com/auca/com.116-118}
            \item[Class Discussions]\hfill\\
                \url{https://piazza.com/class/j6tfld3rvnl50r}
            \item[Place]\hfill\\
                AUCA, room 434\\
                AUCA, laboratory G30, G31
            \item[Time]\hfill\\
                Lecture: Monday 12:45\\
                Lecture: Friday 12:45\\
                Lab: Thursday 10:50\\
                Lab: Thursday 12:45\\
                Lab: Thursday 14:10
        \end{description}

    \section{Contact Information}

        \begin{description}
            \item[Instructor]\hfill\\
                Shostak Dmitrii Grigorievich\\
                \href{mailto:shostak_d@auca.kg}{shostak\_d@auca.kg}\\
                Toksaitov Dmitrii Alexandrovich\\
                \href{mailto:toksaitov_d@auca.kg}{toksaitov\_d@auca.kg}
            \item[Office]\hfill\\
                AUCA, room 315\\
                AUCA, Media Laboratory
            \item[Office Hours]\hfill\\
                Monday 15:25--17:00\\
                Tuesday 15:25--17:00\\
                Wednesday 10:00--17:00\\
                Thursday 15:25--17:00\\
                Friday 15:25--17:00
        \end{description}

    \section{Course Overview}

        This course helps to equip students with basic skills needed for
        structural and object-oriented programming. At the completion of the
        course students should understand fundamental programming concepts such
        as flow control, objects, classes, methods, procedural decomposition,
        inheritance and polymorphism; be able to write simple applications using
        most of the capabilities of the Java programming language and apply
        principles of good programming practices throughout the process.  This
        course is designed for Software Engineering majors and minors.

    \section{Topics Covered}

        \begin{itemize}
            \item Introduction to the Process of Software Development
            \item Selections
            \item Loops
            \item Methods
            \item Single- and Multidimensional Arrays
            \item Objects and Classes
            \item Inheritance and Polymorphism
            \item Abstract Classes and Interfaces
            \item Exception Handling
            \item GUI and Computer Graphics Basics
            \item Generics and Container Classes
            \item Working with I/O
        \end{itemize}

    \section{Exams}

        \subsection{Lectures}

            Students will have to take midterm and final examinations on topics
            discussed during lectures. Each examination is in the form of a quiz
            with a set of open and multiple choice questions.

        \subsection{Labs}

            Students will have 8 laboratory tasks, get a number of problems from
            an Online Judge System, and have to finish two projects developing
            real-world applications. Students will have to defend their work to
            the instructor during separate midterm and final examination
            sessions.

    \section{Reading}

        Introduction to Java Programming, Comprehensive, 8th Edition by Y.
        Daniel Liang (AUCA Library Call Number: QA76.73.J38 L5218 2011, ISBN:
        978-0132130806)

    \section{Grading}

        \subsection{Lectures}

            \begin{itemize}
            	\item Class Participation (through Piazza) (5\%)
                \item Midterm (15\%)
                \item Final (20\%)
            \end{itemize}

        \subsection{Labs}

            \begin{itemize}
                \item Labs 1--4 (10\%)
                \item Online Judge Problems (10\%)
                \item Project \#1 (10\%)\\\\
                    \textbf{Midterm Defense} (Labs + Online Judge Problems + Project \#1)\\
                \item Labs 5--8 (10\%)
                \item Online Judge Problems (10\%)
                \item Project \#2 (10\%)\\\\
                    \textbf{Final Defense} (Labs + Online Judge Problems + Project \#2)\\
            \end{itemize}

        \begin{itemize} \itemsep-10pt \parskip0pt \parsep0pt
            \item[--] 92\%--100\%: A\\
            \item[--] 85\%--91\%: A-\\
            \item[--] 80\%--84\%: B+\\
            \item[--] 75\%--79\%: B\\
            \item[--] 70\%--74\%: B-\\
            \item[--] 65\%--69\%: C+\\
            \item[--] 60\%--64\%: C\\
            \item[--] 55\%--59\%: C-\\
            \item[--] 50\%--54\%: D+\\
            \item[--] 45\%--49\%: D\\
            \item[--] 40\%--44\%: D-\\
            \item[--] Less than 40\%: F
        \end{itemize}

    \section{Rules}

        Students are required to follow the rules of conduct of the Software
        Engineering Department and American University of Central Asia.

        Team work is NOT encouraged. The same blocks of code or similar
        structural pieces in separate works will be considered as academic
        dishonesty and all parties will get zero for the task.

        Attendance is mandatory. Three or more skipped classes without a
        legitimate reason will decrease the final grade by 5\% for each missed
        day.

\end{document}

